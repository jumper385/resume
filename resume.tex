\documentclass{article}[a4page]
\usepackage{babel}
\usepackage[left=0.5in, right=0.5in, bottom=0.75in, top=0.75in]{geometry}
\usepackage{multicol}
\usepackage{hyperref}

\hypersetup{
    colorlinks=true,
    linkcolor=blue,
    filecolor=magenta,      
    urlcolor=cyan,
}

\title{
Curriculum Vitae \\
\small{Email: \href{mailto:henrychen385@gmail.com}{henrychen385@gmail.com}}, 
\small{Mobile: \href{tel:61452611415}{+61 452 611 415} }, \\
\small{Website:  \href{https://digism.xyz}{Digism Design} },
\small{Linkedin: \href{https://www.linkedin.com/in/henry-chen-21b672176/}{Connect with me}},\\
\small{GitHub: \href{https://github.com/jumper385}{Checkout my Repo!}}
}
\author{Henry Jian Chen}
\renewcommand{\abstractname}{Professional Summary}

\begin{document}
\maketitle
\abstract{I’m a UWA Electrical Engineering and Management Student with skills in machine learning, web development \& IT management and mechatronic design. I currently hold a casual position at Vital Trace as an Undergraduate Electronics Engineer. I also work with Engineers Without Borders and UWA Motorsports and as a Project Manager and Electrical Engineer. I was previously involved with INPEX as an Instrumentations \& Controls Engineer and The Noisy Guts Project as a Data Scientist and Biomedical Engineer.}

\section*{Technical Skills}
\subsection*{Electronics Design}
\begin{itemize}
    \item Specializing in STM32 Development, Low Voltage Power Distribution and Analog Design.
    \item Designed and optimized electrochemical DAQ designs for production and RnD purposes at Vitaltrace 
    \item Designed a High Current Power Distribution Module onboard our EV at UWA Motorsports
    \item Developed a Solar Lead-Acid Battery onboard a hobbyist Experimental Aircraft
\end{itemize}

\subsection*{Full Stack Web Development, App Development \& IT Management \hfill}
\begin{itemize}
\item Worked with technologies including Electron, Flutter, React, Express, Socket.IO and Docker for 8 years – building full-stack web and mobile apps.
\item Specializing in creating custom human-machine interfaces for embedded systems.
\item I've developed websites and data collection systems for UWA's University Engineers' Club, Engineers Without Borders and The Noisy Guts Project.
\item Built, Managed and Hosted Websites on virtual private servers.
\end{itemize}

\subsection*{Machine Learning}
\begin{itemize}
\item Worked with Python for four years – cleaning raw data, finding features and building models for classification and regression and natural language processing.
\item Designed and tested sensor drift detection algorithms for use for conditional monitoring on a sensor INPEX.
\item Engineered features from a sparse dataset and applied the abstracted signal to successfully find relationships with other parts of our dataset in the Noisy Guts Project.
\end{itemize}

\subsection*{Oil and Gas Functional Safety - Safety Instrumented Functions (SIF)}
\begin{itemize}
\item Audited and Ensured that Safety Instrumented Functions were Being Tested On-Time
\item Developed Maintenance Work instructions to ensure proper testing is carried out on our SIFs.
\item Developed Software Changes for INPEX's CPF and FPSO Facilities safety systems.
\end{itemize}

\section*{Work Experience}

\subsection*{Vital Trace: Undergraduate Electronics Engineer (Mar 2021 - Present)}
\begin{itemize}
    \item \textbf{Responsibilities: } I am responsible for researching, designing and developing the best possible electronics onboard our DAQ's. I am also responsible for product development, project management, data analysis and creating human-machine interfaces for our DAQ's. 
    \item \textbf{Achievements: } Designed a tool to simulate a sensor's electrochemical response. We set out to design the simplest system possible - thus reducing our work down to good component selection and appropriate PCB layout. This tool was designed to make testing our DAQ's cheaper, simpler and accurate.
        \item Designed electrochemical DAQ's - working on product development and RnD.
        \begin{itemize}
            \item Research and Development: Develop a DAQ platform that we could test filters on - designed to help us test our DAQ filter designs in the real world. I was required to design active filters, ultra-low noise power supplies, digital signal processing algorithms and design an appropriate PCB.
            \item Product Development: Interface with partners in India, Provide them with a prototype legplate schematics, Revise their work and iterate on the product with them
        \end{itemize}
        \item Developed a real-time GUI for our DAQ's - used Electron, Svelte and WebGL Plots
        \item Successfully led and delivered the SuperDAQ Project - A device used to test 5 parallel potentiostat devices.
        \item Designed, built and tested a Lithium-Ion Battery Pack for SuperDAQ - Ensured that the pack would fail safely on all identified failure cases, designed mechanical housing for the pack and interfaced with a contractor to secure the pack to our device.
\end{itemize}

\subsection*{UWA Motorsports: EE Team Member \hfill (Aug 2019 - Present)}
\begin{itemize}
\item \textbf{Responsibilities}: Research and develop solutions to an electric vehicle’s electrical systems such as the vehicle charging, powertrain and safety whilst also using Altium to design custom PCB’s for various parts of the car
\item \textbf{Achievements}: Successfully designed and built an accumulator temperature sensor for battery safety testing and benchmarking
\item My team and I successfully built the car charger by designing a system that changes 3 Phase power into 400V/40A DC
\item Designed and built a discharge rig to safely dissipate 32MJ of energy from our car accumulator in 30-45 min.
\end{itemize}

\subsection*{Engineers Without Borders: Project Lead \hfill (Feb 2019 - Present)}
\begin{itemize}
\item \textbf{Responsibilities}: I am currently the Project Lead and Principal Engineer for UWA's Engineer's Without Borders' Technical Design Team (App Tech). We are currently underway with an autonomous gardening and farming R\&D Project - with the aims of creating an open-source platform to help Aussie farmers make better decisions and manage their risks. 
\item \textbf{Achievements}: Currently co-leading a team of 16 members - taught them to successfully design, manufacture and assemble PCB's. 
\item Designed a LoRa node based on the RN2903 IC. This was brought about due to problems we were having with the RFM95 Module's
\item Currently designing a 2S3P Li-Ion Battery Pack designed to be charged through solar power and placed underground.
\item Led a team of 5 electrical/software engineering students to create a data logging module for remote use in Cambodia.
\item Wrote the original code for the App - created for the team as a baseline version of the app. 
\item We completed and shipped a completed app in January 2020
\end{itemize}

\subsection*{Other Roles}
\emph{INPEX} (Instrumentation and Controls Engineer), \emph{Noisy Guts} (Data Science and App Developer) \emph{Altronics Balcatta} (Sales Assistant), \emph{Austin Computers} (Sales Assistant), \emph{University Engineers’s Club} (Webmaster/Web Development)

\section*{Education}
\subsection*{University of Western Australia \hfill (Feb 2019 - Present)}
\begin{itemize}
\item Awarded \$25,000 Scholarship from UWA Engineering
\item BSc: Electrical Engineering \& Management
\item Master of Professional Engineering Assured Pathway
\end{itemize}
\end{document}